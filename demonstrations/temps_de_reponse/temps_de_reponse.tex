\documentclass[12pt,a4paper]{article} 

\usepackage[a4paper,margin=2cm]{geometry}
\usepackage[T1]{fontenc}
\usepackage{pslatex}
\usepackage[utf8]{inputenc}
\usepackage[greek,francais]{babel} 
\usepackage{graphicx} 
\usepackage{amsmath} 
\setlength{\unitlength}{1mm}
\usepackage{enumitem}
\usepackage{cancel}
\usepackage{amssymb} % pour les ensembles NN
\usepackage{mathrsfs} % pour les hypothèses de récurrences \PP
\usepackage{fancyhdr}
\usepackage[boldLipsian,10pt,GlyphNames]{teubner}
\usepackage{fancybox}
\usepackage{multicol}

\lhead{Antoine Groudiev}
\chead{}
\rhead{MP*, Charlemange}

\lfoot{\jobname.tex}
\cfoot{Temps de réponse}
\rfoot{\thepage}
\renewcommand{\headrulewidth}{0.4pt}
\renewcommand{\footrulewidth}{0.4pt}

\author{Antoine Groudiev}
\title{Temps de réponse maximal}
\date{\today} 

\begin{document}
\maketitle
On modélise le freinage dans le pire cas tel que la vitesse reste constante, égale à $v$, sur une distance $d_f$, puis passant instantanément à $0$.
\section{Constantes}
    On pose :
    \begin{enumerate}
        \item $v$, la vitesse de la voiture, maximale dans le pire cas
        \item $d_f$, la distance de freinage de la voiture, maximale dans le pire cas
        \item $d_d$, la distance de détection du panneau, minimale dans le pire cas
    \end{enumerate}

\section{Inconnues}
    On cherche à déterminer :
    \begin{enumerate}
        \item $\tau_{max}$, le temps aloué pour traiter une image, qu'on cherche à maximiser
        \item $f$, la fréquence de rafraichissement de la détection de panneau, qu'on cherche à minimiser
    \end{enumerate}

    On a la relation :
    $$f = \frac{1}{\tau_{max}}$$
\section{Cahier des charges}
    On veut détecter le panneau de telle sorte que la voiture s'arrête avant le panneau. 
    Ainsi, la distance allouée à l'algorithme pour traiter l'image est de :
    $$d = d_d - d_f$$
    En divisant par $v$ :
    $$\boxed{\tau_{max} = \frac{d_d - d_f}{v}}$$
    on a également :
    $$\boxed{f = \frac{v}{d_d - d_f}}$$

\section{Application numérique}
On utilise les valeurs expérimentales suivantes :
\begin{enumerate}
    \item $v = 25 \, km.h^{-1}$
    \item $d_f = 0,50 \, m$ dans le pire cas
    \item $d_d = 3,1 \, m$
\end{enumerate}
L'application numérique donne $\boxed{\tau_{max} = 0,36 \, s}$ et $\boxed{f = 2.8 \, Hz}$.

\end{document}