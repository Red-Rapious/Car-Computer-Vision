\documentclass[12pt,a4paper]{article} 

\usepackage[a4paper,margin=2cm]{geometry}
\usepackage[T1]{fontenc}
\usepackage{pslatex}
\usepackage[utf8]{inputenc}
\usepackage[greek,french]{babel} 
\usepackage{graphicx} 
\usepackage{amsmath} 
\setlength{\unitlength}{1mm}
\usepackage{enumitem}
\usepackage{cancel}
\usepackage{amssymb} % pour les ensembles NN
\usepackage{mathrsfs} % pour les hypothèses de récurrences \PP
\usepackage{fancyhdr}
\usepackage[boldLipsian,10pt,GlyphNames]{teubner}
\usepackage{fancybox}
\usepackage{multicol}
\usepackage{tikz}

\lhead{Antoine Groudiev}
\chead{}
\rhead{MP*, Charlemange}

\lfoot{\jobname.tex}
\cfoot{Calcul de distance}
\rfoot{\thepage}
\renewcommand{\headrulewidth}{0.4pt}
\renewcommand{\footrulewidth}{0.4pt}

\author{Antoine Groudiev}
\title{Calcul de la distance objet-caméra à l'aide de la mesure expérimentale de la focale de la caméra}
\date{\today} 

\begin{document}
\maketitle

\section{Variables et constantes}
\subsection{Variables}
    \begin{enumerate}
        \item $d$, la taille en pixels de l'objet photographié lors de la mesure de la focale, éventuellement à un facteur de conversion du capteur près (respectivement $d'$ lors de la prise de vue réelle)
        \item $Z$, la distance de la caméra à l'objet lors de la mesure de la focale (respectivement $Z'$ lors de la prise de vue réelle)
    \end{enumerate}
    
    \subsection{Constantes}
    Ces constantes ne varient pas entre la mesure de la focale et la prise de vue réelle.
    \begin{enumerate}
        \item $D$, la taille réelle de l'objet
        \item $f$, la distance focale de la caméra
        \item $\alpha$, le champ du vue de la caméra
    \end{enumerate}
    
\section{Schéma}
    \begin{center}
        \begin{tikzpicture}
            \draw[step=1cm,gray,very thin] (0.01,0.01) grid (12.99,5.99);

            % Axe optique
            \draw [very thick, ->] (0, 3) -- (13, 3) node[anchor=west] {$(\Delta)$};
            
            % Lentille
            \draw [very thick, <->] (8, 0.5) -- (8, 5.5) node[anchor=south] {$(L)$};

            % Objet
            \draw [thick, |-|] (2, 1) -- (2, 5);
            \draw [thick, <->] (1.5, 3) -- (1.5, 5) node [midway, left] {$\frac{D}{2}$};

            \draw [thick, <->] (2, 1) -- (8, 1) node [midway, below] {$Z$};


            % Ecran
            \draw [very thick] (12, 0) -- (12, 5.5) node [anchor=north, above] {$Capteur$};
            \draw [thick, <->] (12.5, 1.66) -- (12.5, 3) node [midway, right] {$\frac{d}{2}$};

            % Focale
            \draw [thick, <->] (8, 1) -- (12, 1) node [midway, below] {$f$};

            % Rayons
            \draw [thick, ->] (1, 5.3) -- (12, 1.66);
            \draw [thick, ->] (1, 0.66) -- (12, 4.33);

            % Angles
            \draw [thick] (5, 3) arc(0:-{atan(1/3)}:-3cm) node[midway,left] {$\alpha$};
            \draw [thick] (11, 3) arc(0:-{atan(1/3)}:3cm) node[midway,right] {$\alpha$};



        \end{tikzpicture}
    \end{center}

\section{Relation entre $f$ et $Z$}
En appliquant la définition de la tangente d'un angle dans un triangle rectangle, il vient :
$$tan(\alpha) = \frac{d}{2} \times \frac{1}{f} = \frac{D}{2} \times \frac{1}{Z}$$

On peut dès lors, en connaissant $D$, $Z$ et $d$, calculer $f$ (formule de la mesure de la focale) :
$$\boxed{f = \frac{Z \times d}{D}}$$

Enfin, puisque $f$ et $D$ sont des constantes on peut obtenir $Z'$ en fonction uniquement de $d'$ (formule de la prise de vue réelle):
$$\boxed{Z' = \frac{f \times D}{d'}}$$

\end{document}