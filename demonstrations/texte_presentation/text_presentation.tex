\documentclass[12pt,a4paper]{article} 

\usepackage[a4paper,margin=2cm]{geometry}
\usepackage[T1]{fontenc}
\usepackage{pslatex}
\usepackage[utf8]{inputenc}
\usepackage[greek,french]{babel} 
\usepackage{graphicx} 
\usepackage{amsmath} 
\setlength{\unitlength}{1mm}
\usepackage{enumitem}
\usepackage{cancel}
\usepackage{amssymb} % pour les ensembles NN
\usepackage{mathrsfs} % pour les hypothèses de récurrences \PP
\usepackage{fancyhdr}
\usepackage[boldLipsian,10pt,GlyphNames]{teubner}
\usepackage{fancybox}
\usepackage{hyperref}

\lhead{Antoine Groudiev}
\chead{}
\rhead{MP*, Charlemange}

\rfoot{\thepage}
\renewcommand{\headrulewidth}{0.4pt}
\renewcommand{\footrulewidth}{0.4pt}

\author{Antoine Groudiev}
\title{\textit{Présentation orale du TIPE :} \\ Vision par ordinateur appliquée aux véhicules autonomes}
\date{\today} 

\begin{document}
\maketitle



\section*{Introduction : Définition des objectifs et contraintes}
    Bonjour, je vais vous présenter mon travail d'initiative personnelle encadré, sur le thème de la ville. La ville est par nature un lieu de transport ; et l'on s'imagine souvent les transports du futur comme autonomes, c'est à dire capables d'interagir avec leur environnement sans action de l'utilisateur. \\

    Avec mon collègue Mathieu Spiegel, nous avons donc entrepris de modéliser un véhicule autonome à échelle réduite, pour étudier la faisabilité d'une telle innovation. \\

    Le robot est notamment équipé d'un Raspberry Pi, un micro-ordinateur, de moteurs et d'une caméra. \\

    Le projet a été séparé un deux parties distinctes mais complémentaires : une partie de construction, mécanique et électronique du robot, par mon camarade ; et une partie logicielle, axée sur l'implémentation de vision par ordinateur. \\

    La vision par ordinateur vise à traiter un flux vidéo pour en extraire des informations utiles, comme la présence ou non d'un panneau stop, afin de pouvoir agir un conséquence. \\

    Après avoir défini les objectifs et contraintes, j'ai comparé deux algorithmes à l'aide d'une bibliothèque externe, puis j'ai implémenté de zéro l'algorithme de Viola et Jones.

\section{Haut niveau : utilisation de bibliothèques}

    Deux contraintes sont décisives dans ce projet : la première, celle d'exactitude, nécessite de ne pas omettre d'objets à détecter, pour être sûr de s'arrêter à une intersection par exemple. La seconde, celle de l'efficaité, est notamment liée à la nature embarquée du processeur. \\

    On voit en effet sur ce graphe que j'ai réalisé que la Rapsberry Pi est beaucoup plus lente que de nombreux ordinateurs modernes, nécessitant donc des algorithmes efficaces. \\

    La problématique au coeur de mon travail a donc été de réussir à allier précision et efficacité de la détection d'objet dans une image. \\

    J'ai commencé par comparer deux algorithmes de vision par ordinateur à l'aide de 


\subsection{Algorithme de Viola-Jones}


\subsection{Algorithme YOLO}


\subsection{Calcul de la distance caméra-objet}



\section{Bas niveau : implémentation  personnelle de l’algorithme de Viola-Jones}


\subsection{Implémentation et explication approfondie de Viola-Jones}


\subsection{Traitement des images d’entraînement}


\subsection{Tests et résultats}


\section{Conclusion}


\begin{thebibliography}{9}
    \bibitem{}Paul Viola, Michael Jones, Rapid Object \textit{Detection using a Boosted Cascade of Simple Features}, Conference on Computer Vision, (2001)
    \bibitem{}Anmol Parande, \textit{Understanding and Implementing the Viola-Jones Image Classification Algorithm}, Medium, DataDrivenInvestors, (2019)
    \bibitem{}David Garbage, \textit{Finding distance from camera to object of known size}, StackOverflow, (2011)
    %\bibitem{}Computerphile, \textit{Detecting Faces (Viola Jones Algorithm)}, Computerphile - YouTube
    %\bibitem{}Hmrishav Bandyopadhyay, \textit{YOLO: Real-Time Object Detection Explained}, \url{v7labs.com/blog/yolo-object-detection}

\end{thebibliography}

\section*{Annexes}
\begin{itemize}
    \item Code source complet de mon implémentation de l'algorithme de Viola-Jones
    \item Divers scripts de \textit{benchmark}, utilitaires, de \textit{data processing}, etc.
    \item Démonstration rigoureuse de la distance caméra-objet
    %\item Démonstration du temps de réponse maximal de l'algorithme pour respecter le cahier des charges
\end{itemize}

\end{document}